\chapter{Einleitung}
\section{Ist-Situation}
Im Unternehmen Augenoptik Aigner sind zur Zeit nur wenige Rechner im Einsatz, welche alle mit dem Betriebssystem “Windows 95” funktionieren. Auf diesen existiert ein DOS-Programm welches Kunden,  Aufträge, Lieferanten und lagernde Produkte  verwaltet. Weil mehrere Rechner im Einsatz sind, muss die aktuelle Version des Programms immer auf den Rechner kopiert werden, auf dem man dann gerne arbeiten würde. Dies ist natürlich sehr umständlich und zeitaufwendig, weshalb es Zeit wird, alle Rechner auf ein aktuelles Betriebssystem zu aktualisieren und eine zentrale Datenbank einzurichten, damit man auf allen Rechnern synchronisiert arbeiten kann.\newline
Die Website des Unternehmens Augenoptik Aigner ist aktuell weder responsive noch ist sie visuell ansprechend außerdem ist sie in der Funktionalität beschränkt. Die einzigen Funktionen die man hat sind allgemeine Daten anzusehen, wie die Geschäftszeiten, den Standort des Unternehmens, ein Impressum und man kann den Verkäufer noch mittels eines Formulars kontaktieren. 

\section{Zielsetzung}
In der Verwaltungssoftware sollen alle Kunden, Aufträge, Lieferanten und Brillenfassungen verwaltet werden. Dabei soll es möglich sein, neue Datensätze anzulegen und bestehende zu bearbeiten und wieder zu löschen. Bei den Aufträgen wird zwischen Brillen- und Kontaktlinsenaufträgen unterschieden. Jeder Auftrag soll eine implementierte Preisberechnung und Details zur Glasverarbeitung beinhalten. Zu jedem Auftrag sollen Auftragsbestätigung und Rechnung als Word-Dokument exportiert werden können. \newline Weitere Features sollen sein, Statistiken über die verkauften Brillen oder Kontaktlinsen ansehen zu können und Massen- sowie Einzelnachrichten versenden zu können. Diese Nachrichten sollen entweder als SMS oder als E-Mail werden können. \newline Die Software soll als Desktopanwendung realisiert werden und soll auf eine zentrale Datenbank zugreifen, mit der alle Rechner im Unternehmen Augenoptik Aigner verbunden werden. Dadurch sollen alle Rechner des Unternehmens synchronisiert arbeiten können.
Es ist eine mobile responsive Website zu entwickeln. Auf dieser sollen neben der Lage und den Öffnungszeiten auch die verschiedenen Brillenmodelle angezeigt werden. Der User soll über die Website dem Verkäufer sein Kaufinteresse mitteilen. Bei allgmeinen Fragen müssen User den Verkäufer kontaktieren können. Der Verkäufer kann sich auf der Website anmelden und, dann seine Tabellen ansehen und bearbeiten um zum Beispiel von zu Hause eine neue Brille einzutragen. 

\section{Produkteinsatz und Benutzer}
Das Verwaltungsprogramm wird nur im Unternehmen Augenoptik Aigner eingesetzt und wurde auch auf deren Anforderungen personalisiert. Die Anzahl der Benutzer des Verwaltungsprogrammes beschränken sich auf zwei, da das Unternehmen nicht mehr Mitarbeiter hat.
Da die Website online gestellt wird kann sie von jedem benutzt werden der sie aufruft



